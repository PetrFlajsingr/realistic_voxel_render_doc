% Tento soubor nahraďte vlastním souborem s přílohami (nadpisy níže jsou pouze pro příklad)

% Umístění obsahu paměťového média do příloh je vhodné konzultovat s vedoucím
%\chapter{Obsah přiloženého paměťového média}

%\chapter{Manuál}

%\chapter{Konfigurační soubor}

%\chapter{RelaxNG Schéma konfiguračního souboru}

%\chapter{Plakát}

\chapter{Obsah přiloženého paměťového média}

\begin{itemize}  
\item \texttt{README.md} - Obsahuje informace o aplikaci, jak ji přeložit a používat
\item \texttt{doc/} - Složka obsahující dokumentaci
\begin{itemize}
\item \texttt{thesis/} - Text a zdrojové soubory této práce
\item \texttt{Thesis.pdf} - Vygenerovaná dokumentace 
\end{itemize}
\item \texttt{src/} - Zdrojové soubory
\item \texttt{bin/} - Složka se spustitelným souborem pro systém Linux
\item \texttt{video/} - Složka s prezentačním videem
\end{itemize}

\chapter{Konfigurační soubor}
\label{appendix:configfile}

\begin{lstlisting}[language=toml]
[rendering.compute]
local_size_x = <int> # doporuceno 8
local_size_y = <int> # doporuceno 8

[resources]
path_models = <path string> # cesta do slozky assets/vox v adresari projektu
path_shaders = <path string> # cesta do slozky src/shaders v adresari projektu

[ui.imgui]
path_icons = <path string> # cesta do slozky assets/icons v adresari projektu

# v teto sekci se nachazi take aplikaci
# vygenerovana konfigurace hodnot uzivatelskeho rozhrani

[ui.window]
height = <int>
width = <int>

\end{lstlisting}